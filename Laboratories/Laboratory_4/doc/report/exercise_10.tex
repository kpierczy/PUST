\section{Trajektorie zadane}

W~ramach panelu operatorskiego ziamplementowana zosta�a zak�adka pozwalaj�c zmienia� trajektori� zadan� wszystkich trzech wyj�� procesu w~trzech przedzia�ach czasowych r�wnej d�ugo�ci \footnote{Opis z~perspektywy u�ytkownika zamieszczony zosta� w~sekcji 7}. Tak jak w przypadku stanowiska G-C implementacja sprowadza si� do blok instrukcji warunkowych �ledz�cych warto�� czasu \verb traj. Gdy znajduje si� on w~pwnych ramach warto�ci sterowania ustawiane s� na te widoczne w~paneli HMI. Fragment kodu odpowiedzialny za automat stan�w przedstawiony zosta� na poni�szym listingu.

\vskip 0.5cm
\begin{lstlisting}[style=customc,frame=single] 
    IF WATER_TRAJ_ON THEN
        
        IF WATER_DURATION <> WATER_DURATION_HMI THEN
            WATER_DURATION := WATER_DURATION_HMI;
            traj := 0;
        END_IF;
        
        IF  traj <= 1*WATER_DURATION_HMI-1 THEN
            PID1.SV := WATER_Y1_TRAJ1;
            PID2.SV := WATER_Y2_TRAJ1;
            PID3.SV := WATER_Y3_TRAJ1;
            WATER_CURRENT_TRAJ1 := TRUE;
            WATER_CURRENT_TRAJ2 := FALSE;
            WATER_CURRENT_TRAJ3 := FALSE;
        ELSIF traj <= 2*WATER_DURATION_HMI-1 THEN
            PID1.SV := WATER_Y1_TRAJ2;
            PID2.SV := WATER_Y2_TRAJ2;
            PID3.SV := WATER_Y3_TRAJ2;
            WATER_CURRENT_TRAJ1 := FALSE;
            WATER_CURRENT_TRAJ2 := TRUE;
            WATER_CURRENT_TRAJ3 := FALSE;
        ELSIF traj <= 3*WATER_DURATION_HMI-1 THEN
            PID1.SV := WATER_Y1_TRAJ3;
            PID2.SV := WATER_Y2_TRAJ3;
            PID3.SV := WATER_Y3_TRAJ3;
            WATER_CURRENT_TRAJ1 := FALSE;
            WATER_CURRENT_TRAJ2 := FALSE;
            WATER_CURRENT_TRAJ3 := TRUE;
        ELSIF traj >= 3*WATER_DURATION_HMI-1 THEN
            PID1.SV := WATER_Y1_TRAJ1;
            PID2.SV := WATER_Y2_TRAJ1;
            PID3.SV := WATER_Y3_TRAJ1;
            WATER_CURRENT_TRAJ1 := TRUE;
            WATER_CURRENT_TRAJ2 := FALSE;
            WATER_CURRENT_TRAJ3 := FALSE;
            traj :=0;
        ELSE;
        END_IF;
        traj := traj + 1;
    ELSE
        WATER_CURRENT_TRAJ1 := FALSE;
        WATER_CURRENT_TRAJ2 := FALSE;
        WATER_CURRENT_TRAJ3 := FALSE;
        traj := 0;
    END_IF;
\end{lstlisting}
\vskip 1 cm

Zmienne \verb WATER_CURRENT_TRAJx  odpowiedzialne s� za wizualizacj� aktualnego stanu uk�adu. Po przej�ciu automatu przez wszystkie stany jest on resetowany, a~ca�a sekwencja powtarza si�.