Sprecyzowany schemat post�powania powoli� nam obiektywnie por�wnywa� poszczeg�lne warianty i~znacznie przyspieszy� prowadzenie bada�. Na tym etapie warto podkre�li�, �e jako \textit{zmienna rozmywaj�ca} wybrana zosta� warto�� wyj�ciowa obiektu. Przeprowadzone eksprytmenty pokaza�y, �e prowadzi to do poprawienia jako�ci regulacji, co da�o si� przewidzie� ju� na etapie analizy charakterystyki statycznej obiektu. Dalsze badania potwierdzi�y stawian� tez�, co przedstawia poni�szy wykres. Pewien niepok�j mo�e powodowa� fakt zwi�kszenia b��du przy 4 regulatorach DMC, jednak by�o to spowodowane nieoptymalnym rozlokowaniem regulator�w na danym zakresie sterowania (regulatory rozlokowane by�y w r�wnych odst�pach).

\vskip 1cm
\begin{figure}[H]
        \begin{tikzpicture}
            \begin{axis}[
                    width=0.20\textwidth,
                    height=0.1\textheight,
                    xmin=0,xmax=6,ymin=60,ymax=100,
                    xlabel={liczba regulator�w},
                    ylabel={error},
                    xtick={0,1,2,3,4,5,6},
                    legend pos=south east,
                    legend style={font=\tiny},
                    y tick label style={/pgf/number format/1000 sep={,}},
                ]
                
                \addplot[mark=o,draw=blue]               file {data/DMC_error.txt};
                \addplot[mark=o,draw=red]               file {data/PID_error.txt};

                \legend{DMC error, PID error}
            \end{axis}
        \end{tikzpicture}
        \label{DMC_PIS_error}
\end{figure}
\vskip 1cm