Logika rozmyta zosta�a wprowadzona do powszechnego obiegu w~latach 60' ubieg�ego wieku, chocia� wcze�niej zajmowali si� ni� takie osobisto�ci jak prof. Jan �ukasiewicz czy prof. Alfred Tarski. Od tamtej pory znalaz�a ona zastosowania w~szeregu r�nych dziedzin poczynaj�c od analizy mowy, przez podejmowanie decyzji medycznych czy biznesowych, na autonomicznych systemach obronnych ko�cz�c. Swoje miejsce zaj�a tak�e w~dziedzinie szeroko poj�tego sterowania.

Wykorzystanie logiki rozmytej pozwala nie tylko modelowa� procesy nieliniowe z~pomoc� kilku dobrze znanych, liniowych modeli, ale umo�liwia tak�e matematyczne uj�cie nieprecyzyjnej ("rozmytej") wiedzy eksperckiej. Po��czona z~innymi, dobrze przeanalizowanymi na przestrzeni dekad, metodami regulacji stanowi ona pot�ne narz�dzie w~r�kach wsp�czesnych in�ynier�w. Jej przyk�ad pokazuje, �e na r�wni z~post�puj�cym rozwojem technologii, wa�ne jest rozwijanie solidnych podstaw matematycznych, poniewa� to one zapewniaj� nam wydajne wykorzystanie posiadanych zaspob�w obliczeniowych.