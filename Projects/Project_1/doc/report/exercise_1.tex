\chapter{Badanie poprawno�ci punktu pracy}

Pierwsze zadanie sprowadza�o si� do zweryfikowania poprawno�ci, podanego w~tre�ci zadania, punktu pracy $U_{pp}=\num{1.7}$, $Y_{pp}=\num{2}$. Weryfikacja polega�a na podaniu na wej�cie obiektu sta�ej warto�ci pobudzenia $U=\num{1.7}$ przy jednoczesnym za�o�eniu, �e warto�� wyj�cia w~przesz�o�ci ustalona by�a na poziomie $Y=\num{2.0}$ i sprawdzeniu, czy warto�� ta ulegnie zmianie. Wynik eksperymentu zosta� przedstawiony na rys. \ref{work_point_check}. Zgodnie z~danymi podanymi w~tre�ci zadania, punkt $U=\num{1.7},Y=\num{2}$ jest rzeczywi�cie punktem pracy.

\vskip 0.5cm
\begin{figure}[h]
    \centering
        \begin{tikzpicture}
            \begin{axis}[
                    width=0.5\textwidth,
                    xmin=0,xmax=100,ymin=0,ymax=3,
                    xlabel={$t [s]$},
                    ylabel={$y(t)$},
                    xtick={0, 25, 50, 75, 100},
                    ytick={0,0.5,1.0,1.5, 2.0, 2.5, 3.0},
                    legend pos=south east,
                    y tick label style={/pgf/number format/1000 sep={,}},
                ]
                \addplot[const plot, blue]                file {data/exercise_1/work_point_check.txt};
                \legend{$y(t)$}
            \end{axis}
        \end{tikzpicture}
    \caption{Przebieg warto�ci wyj�ciowej procesu po podaniu sta�ego wej�cia $U=\num{1.7}$}
    \label{work_point_check}
\end{figure}
\vskip 0.5cm
