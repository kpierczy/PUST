\chapter{Badanie poprawno�ci punktu pracy}
Zgodnie z tradycj� pierwsze z~zada� polega�o na zweryfikowaniu podanego w tre�ci zadania punktu pracy obiektu regulacji. Punkt ten powinien mie� wsp�rz�dne $u = y = 0$, gdzie kolejne symbole oznaczaj� warto�� sterowania oraz warto�� wyj�ciow�. Weryfikacja polega�a na podaniu na wej�cie procesu warto�ci zerowej i~obserwowaniu zachowania wyj�cia. Zgodnie z oczekiwaniami, wyj�cie ustali�o si� na poziomie $y = 0$, co zosta�o zaprezentowane na wykresie 2.1..

\vskip 0.5cm
\begin{figure}[h]
    \centering
        \begin{tikzpicture}
            \begin{axis}[
                width=0.5\textwidth,
                xmin=0,xmax=100,ymin=-1,ymax=1,
                xlabel={$k$},
                ylabel={$y(k)$},
                xtick={0,50,100},
                ytick={-1,0,1},
                legend pos=south east,
                y tick label style={/pgf/number format/1000 sep=},
                ]
                \addplot[red , semithick ] file { data/exercise_1/work_point_check.txt};
                \legend{y(k)}
            \end{axis}
        \end{tikzpicture}
        \label{work_point_verification}
        \caption{Punkt pracy}
    \end{figure}
\vskip 0.5cm