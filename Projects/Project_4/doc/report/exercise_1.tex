\chapter{Sprawdzenie punktu pracy}

Przed rozpocz�ciem eksperyment�w musimy ustali� dozwolony punkt pracy. Zgodnie z tre�ci� projektu jest to punkt dla sterowania U = [0,0,0,0] ( taki zapis oznacza, �e U1 = 0, U2 = 0, U3 = 0 i U4 = 0 ) i wyj�cie obiektu powinno si� stabilizowa� na warto�ci Y = [0,0,0] ( zapis analogiczny jak w przypadku sterowania).




\vskip 0.5cm
    \begin{figure}[H]
        \begin{tikzpicture}
            \begin{groupplot}[
                group style={group size=1 by 2},
                width=1\textwidth,
                height=0.3\textheight
                ]

                \nextgroupplot
                [xmin=0,xmax=200,ymin=-5,ymax=5,
                xlabel={$t[s]$},ylabel={$Y1,Y2,Y3$},legend cell align=left,
                legend pos=south east]
                \addplot[const plot,color=blue,semithick] file {data/exercise_1/Y1_U1_0_U2_0_U3_0_U4_0.txt};
                \addplot[const plot,color=green,semithick] file {data/exercise_1/Y2_U1_0_U2_0_U3_0_U4_0.txt};
                \addplot[const plot,color=yellow,semithick] file {data/exercise_1/Y3_U1_0_U2_0_U3_0_U4_0.txt};
                \legend{$Y1$,$Y2$,$Y3$}
                
                \nextgroupplot
                [xmin=0,xmax=200,ymin=-5,ymax=5,
                xlabel={$t[s]$},ylabel={$U1,U2,U3,U4$},legend cell align=left,
                legend pos=north east]
                \addplot[const plot,color=blue,semithick] file {data/exercise_1/U1.txt};
                \addplot[const plot,color=green,semithick] file {data/exercise_1/U2.txt};
                \addplot[const plot,color=yellow,semithick] file {data/exercise_1/U3.txt};
                \addplot[const plot,color=red,semithick] file {data/exercise_1/U4.txt};
                \legend{$U1$,$U2$,$U3$,$U4$}
            
            \end{groupplot}
        \end{tikzpicture}
        \label{step_response}
        \caption{Punkt pracy}
    \end{figure}
\vskip 0.5cm



Jak widzimy podany punkt pracy w projekcie jest poprawny i obiekt si� stabilizuje na podanych warto�ciach.
