\chapter{Stanowisko Inteco}

Po zrealizowaniu ostatniego projektu z~Projektowania Uk�ad�w Sterowania ka�dy student Automatyki i~Robotyki mo�e stwierdzi�, �e stanowisko grzej�co ch�odz�ce nie ma przed nim �adnych tajemnic. Je�eli kiedykolwiek zostanie przed kim� z~nas postawiony obiekt, to regulator PID lub DMC zaprojektuje i~nastroi on intuicyjnie, niemal bezwiednie nawet je�eli nie b�dzie to potrzebne. �wiadomi takiego stanu rzeczy przechodzimy do konfrontacji z~kolejnym obiektem --- stanowiskiem Inteco.

Jak ju� opisano w~przypadku stanowiska G-C awaria czujnik�w jest jedn� z~potencjalnie najniebezpieczniejszych awarii w~systemie. Pami�taj�c o~tym tak�e w~przypadku stanowiska Inteco zaimplementowali�my prosty mechanizm zabezpieczaj�cy analogiczny do opisanego wy�ej. Po wykryciu warto�ci krytycznej ($\geq$19.95) na kt�rymkolwiek z~czujnik�w poziomu zapalana jest odpowiednia flaga. Sprawdzana jest ona z~kolei w~momencie aktualizacji sterowania. Je�eli jest zapalona, to koresponduj�cy z~ni� zaw�r zostaje w~pe�ni otwarty. W~przeciwnym wypadku aplikowane jest sterowanie wynikaj�ce z~regulatora. Blok instrukcji warunkowych odpowiedzialnych za realizacj� mechanizmu zabezpieczaj�cego zosta� przedsatawiony na listingu poni�ej.

\vskip 1cm
\begin{lstlisting}[style=customc,frame=single] 
    IF WATER_Y1 > 19.95 THEN
        WATER_Y1_OVERFLOW := TRUE;
    ELSE
        WATER_Y1_OVERFLOW := FALSE;
    END_IF;
    IF WATER_Y2 > 19.95 THEN
        WATER_Y2_OVERFLOW := TRUE;
        ELSE
        WATER_Y2_OVERFLOW := FALSE;
    END_IF;
    IF WATER_Y3 > 19.95 THEN
        WATER_Y3_OVERFLOW := TRUE;
        ELSE
        WATER_Y3_OVERFLOW := FALSE;
    END_IF;
\end{lstlisting}
\vskip 1cm

