
\begin{figure}[H]
    \centering
    \subfloat{
		\begin{tikzpicture}
			\begin{axis}[
				width=0.76\textwidth,
				height=0.45\textwidth,
				xmin=0,xmax=200,ymin=10,ymax=15,
				xlabel={$t [s]$},
				ylabel={$y(t)$},
				xtick={0, 40, 80, 120, 160, 200},
				ytick={10, 11, 12, 13, 14, 15},
				legend pos=north west,
				y tick label style={/pgf/number format/1000 sep={,}},
			]
			\addplot[const plot, blue]   file {data/exercise_8/pid_builtin/Y1.txt};
			\addplot[const plot, red]    file {data/exercise_8/pid_builtin/Y1_zad.txt};
			\legend{$Y1$, $Y1_zad$}
		\end{axis}
		\end{tikzpicture}
	}
    \vskip 0.5cm
    \subfloat{
		\begin{tikzpicture}
			\begin{axis}[
				width=0.76\textwidth,
				height=0.45\textwidth,
				xmin=0,xmax=200,ymin=10,ymax=15,
				xlabel={$t [s]$},
				ylabel={$y(t)$},
				xtick={0, 40, 80, 120, 160, 200},
				ytick={10, 11, 12, 13, 14, 15},
				legend pos=north west,
				y tick label style={/pgf/number format/1000 sep={,}},
			]
			\addplot[const plot, blue]   file {data/exercise_8/pid_builtin/Y2.txt};
			\addplot[const plot, red]    file {data/exercise_8/pid_builtin/Y2_zad.txt};
			\legend{$Y2$, $Y2_zad$}
		\end{axis}
		\end{tikzpicture}
	}
	\vskip 0.5cm
    \subfloat{
		\begin{tikzpicture}
			\begin{axis}[
				width=0.76\textwidth,
				height=0.45\textwidth,
				xmin=0,xmax=200,ymin=10,ymax=15,
				xlabel={$t [s]$},
				ylabel={$y(t)$},
				xtick={0, 40, 80, 120, 160, 200},
				ytick={10, 11, 12, 13, 14, 15},
				legend pos=north west,
				y tick label style={/pgf/number format/1000 sep={,}},
			]
			\addplot[const plot, blue]   file {data/exercise_8/pid_builtin/Y3.txt};
			\addplot[const plot, red]    file {data/exercise_8/pid_builtin/Y3_zad.txt};
			\legend{$Y3$, $Y3_zad$}
		\end{axis}
		\end{tikzpicture}
	}
    \vskip 0.5cm

    \caption{Przebiegi sygna��w wyj�ciowych przy ostatecznej wersji wbudowanego reulatora PID}
    \label{definit_pid_builtin_y}
\end{figure}

\begin{figure}[H]
    \centering
	\subfloat{
		\begin{tikzpicture}
			\begin{axis}[
				width=0.76\textwidth,
				height=0.45\textwidth,
				xmin=0,xmax=200,ymin=-0.2,ymax=1.2,
				xlabel={$t [s]$},
				ylabel={$y(t)$},
				xtick={0, 40, 80, 120, 160, 200},
				ytick={-0.2, 0, 0.2, 0.4, 0.6, 0.8, 1.0, 1.2},
				legend pos=north west,
				y tick label style={/pgf/number format/1000 sep={,}},
			]
			\addplot[const plot, blue]   file {data/exercise_8/pid_builtin/U1.txt};
			\legend{$U1$}
		\end{axis}
		\end{tikzpicture}
	}
	\vskip 0.5cm
	\subfloat{
		\begin{tikzpicture}
			\begin{axis}[
				width=0.76\textwidth,
				height=0.45\textwidth,
				xmin=0,xmax=200,ymin=-0.2,ymax=1.2,
				xlabel={$t [s]$},
				ylabel={$y(t)$},
				xtick={0, 40, 80, 120, 160, 200},
				ytick={-0.2, 0, 0.2, 0.4, 0.6, 0.8, 1.0, 1.2},
				legend pos=north west,
				y tick label style={/pgf/number format/1000 sep={,}},
			]
			\addplot[const plot, blue]   file {data/exercise_8/pid_builtin/U2.txt};
			\legend{$U2$}
		\end{axis}
		\end{tikzpicture}
	}
	\vskip 0.5cm
	\subfloat{
		\begin{tikzpicture}
			\begin{axis}[
				width=0.76\textwidth,
				height=0.45\textwidth,
				xmin=0,xmax=200,ymin=-0.2,ymax=1.2,
				xlabel={$t [s]$},
				ylabel={$y(t)$},
				xtick={0, 40, 80, 120, 160, 200},
				ytick={-0.2, 0, 0.2, 0.4, 0.6, 0.8, 1.0, 1.2},
				legend pos=north west,
				y tick label style={/pgf/number format/1000 sep={,}},
			]
			\addplot[const plot, blue]   file {data/exercise_8/pid_builtin/U3.txt};
			\legend{$U3$}
		\end{axis}
		\end{tikzpicture}
	}
    \vskip 0.5cm

    \caption{Przebiegi sygna��w steruj�cych przy ostatecznej wersji wbudowanego reulatora PID}
    \label{definit_pid_builtin_u}
\end{figure}



\section{Wbudowany regulator PID}

�rodowisko \textit{GX Works} posiada wbudowany blok funkcyjny realizuj�cy regulator PID. Kosztem zupe�nie nieintuicyjnego interfejsu niweluje ona potrzeb� budowania jego w�asnej implementacji. W~ramach projektu przetestowali�my jego dzia�anie w~tych samych warunkach, w kt�rych testowany by� nasz autorski regulator. Procedura strojenia by�a identyczna z~t� opisan� w~poprzedniej sekcji. Wyniki pracy nastrojonego regulatora przedstawia rys. \ref{definit_PID_builtin_y} i~\ref{definit_PID_builtin_u}.

Nastawy znalezione dla regulatora wbudowanego s� identyczne jak w~przypadku wersji autorskie. Przy takim ustawieniu przebiegi warto�ci wyj�ciowych s� r�wnie� identyczne, przez co wszystkie wniosku z~poprzedniego paragrafu pozostaj� w~mocy. Jedyn�, potencjalnie znacz�c�, r�nic� mi�dzy regulatorami jest spos�b zarz�dzania sygna�ami steruj�cymi. Jak wida� na rys. \ref{definit_PID_builtin_u} wbudowany regulator steruje za pomoc� \textbf{g�sto�ci sygna��w binarnych}. Podej�cie takie jest jak najbardziej do przyj�cia w~przypadku element� wykonawczych o~du�ej bezw�adno�ci. W~przypadku element�w "szybszych" \underline{mo�e} to doprowadzi� do ich uszkodzenia.